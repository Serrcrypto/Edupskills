\newcounter{nP}
\newcounter{nS}
\newcommand{\pp}[1]{\addtocounter{nP}{1}
\paragraph{\arabic{nS}.\arabic{nP} #1}}

\documentclass[12pt]{article}
\usepackage{amsmath}
\usepackage[left=4cm,top=3cm,right=3cm, bottom=2cm]{geometry}%\def\I{{\'{\i}}}
\usepackage[dvips]{epsfig}
\usepackage{picinpar}
\usepackage{graphicx}

\usepackage[spanish]{babel}
\usepackage[latin1]{inputenc}
\usepackage{color}
\usepackage{amsfonts}
\usepackage{amssymb}
\usepackage{epsfig}
\usepackage{graphics}
\newtheorem{thm}{Teorema}
\usepackage{array}

\usepackage{graphicx} % graficos
\usepackage{multirow, array} % para las tablas
\usepackage{float} % para usar [H]

\usepackage{float}
\usepackage{longtable}
\usepackage{capt-of}

\begin{document}
%\By Adilene Palma-Asunción
\setlength{\unitlength}{1 cm} %Especificar unidad de trabajo
\thispagestyle{empty}
\begin{center}
\textbf{{\Huge }\\[0.5cm]
{\LARGE }\\[1.25cm]
{\Large }\\[1cm]
\begin{figure} [h]
\centering
\scalebox{0.5}{\includegraphics[scale=1.7]{Edupskills_logo_dark_transparent.png}}
\end{figure}
{\LARGE {Edupskills Activities}} \\[0.2cm]
{ {Verification method of proof}}\\[1cm]
{\Large User ID/name}\\[0.2cm]
{\large {Subjects applications}} \\
\\[0.2cm]
\ { Available loading date}}
\end{center}

\newpage

\centerline{\Large\bf Edupskills dynamics construction} \vskip 0.3cm
%report
 About this is a template to follow to incorporate information on practices activities to increase the abilities focus to Know and incorporate into educational planning.

%\centerline{\large\bf Subject: Robotics Introduction}
%\vspace{0.25cm}
%\centerline{\large Assertiveness}
%\begin{itemize}
%    \item Introduction to Robotics: Definition Industrial Robots
    %\item Motivation to know about Robotics tecnology
%    \item Practical application
%    \item Simulation resources/robotic platform (optional)
%    \item Method to application (Focus to do)
%    \item Conclusions  
%\end{itemize}
%\vspace{0.25cm}

\centerline{\large Abstract} \vskip 0.3cm
This is not only to memorize the concept of an robot. In fact, it is to understand and put the learned concept into practical life focused on doing achieve top performance: Real Learning.

%centerline{\large Instructions:}
%This activity is limited for time and specific targets to %follows the following three steps:

%\begin{itemize}
 %   \item \bf{Step 1.} Information analysis (To known)
  %      \item \bf{Step 2.} Development and practical execution
   %         \item \bf{Step 3.} Evaluation rubric
%\end{itemize}

%\vspace{2em}
%\begin{bf}

%\end{bf}

%\section{Goals}


%\section{Problem statement}

\section{Body}
Your main points should be clear and concise. You do not want your audience wondering about what you are talking.

%\begin{itemize}
 %   \item Transitions can help keep your speech clear. Between every main point.
%    \item  A transition should contain a summary statement of the concept you have just talked about. 
%    \item Then you should show how the topic you just spoke of is related to the next topic. 
% \end{itemize}

%\begin{tabular}{| c | c |}
%\hline
%Fruta & Cantidad \\ \hline
%Manzana & 4 \\
%Naranja & 10 \\
%Plátano & 3 \\ \hline
%\end{tabular}

\begin{table}[H]
\begin{center}
%\begin{tabular}{| c | c | c | c | }
\begin{tabular}{p{3.5cm} p{10cm}}
\hline
\multicolumn{2}{ |c| }
{\large \bf{Edupskills Dynamics Construction}} \\ \hline
  \bf{Application area/Topic}      &  General descriptions   \\ \hline
\bf{Subject} & Name of subject  \\ \hline
\bf{Background} & Prior knowledge \\ \hline
\bf{Theoretical framework of dynamics} &  Ex. leadership, comunications, teamwork... \\ \hline
\bf{Total number of participants} & Number of people for which the dynamic is applicable \\ \hline 
\bf{Profile of participants} &  Participant profiles are used by the practical activities to verify that the recruited participants fit the segment that is intended to be applicable. \\ \hline
\bf{Main goals}  & Something that you hope to achieve in the future with the pplicaiton of activity. \\ \hline
\bf{Time to application} & A real-time application is an application that functions within a time frame that the user senses as immediate or current. \\ \hline
\bf{Specific Characteristics to dynamics} & A specific characteristic of competition clearly exists, a specific characteristic relating to content, to creation and which also relates to aplicables activities. \\ \hline
 \bf{Characteristics of the place for practical activity} & Outdoors or indoors, specific the place that the activity needs. \\ \hline
\bf{Practice Props}  & Elements to support something physically, often by leaning it against something else or putting something under activities. Simulation resources/physical platform (optional). \\ \hline
\bf{Dynamics description}  & 

\centerline{ Instructions:}
This activity is limited for time and specific targets to follows the following three steps:

\begin{itemize}
    \item \bf{Step 1.} Information analysis (To known)
        \item \bf{Step 2.} Development and practical execution (To do)
            \item \bf{Step 3.} Rules and restrictions
                \item \bf{Step 4.} Evaluation rubric (To be)
\end{itemize}
\\ \hline
\bf{Doubts and questions (Debriefing)} & If you have doubt or doubts about something, you feel uncertain about it and do not know whether it is true or possible, to focus to student to solve. \\  \hline
\bf{Digital evidence} & Photos, video, quiz , etc. \\ \hline
\bf{Experience} & Describe unsuccessful cases and the problems to be implemented that they identify and could be prevented. \\ \hline
%\bf{Doubts and questions} \\ \hline
\end{tabular}
%\caption{Edupskills dynamics}
\label{tab:coches}
\end{center}
\end{table}

\section{The last}

A review reminds your apprentices what you have just talked about. In the review, you get a chance to repeat the important parts of your speech that the apprenticeship should keep in mind.
The clincher includes any final thoughts you want to leave with your audience. 

\vskip 0.2cm
\centering :)

\bibliographystyle{ieeetr}
%\longrightarrow\bibliographystyle{unsrt}
%\bibliography{librero}
\end{document}

%By Adilene Palma
%***********************


